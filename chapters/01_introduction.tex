\pagelayout{wide}
\chapter*{Introduction\;}
\labch{intro}

The invention of the automobile in the late 19th century revolutionized human mobility, enabling unprecedented freedom to traverse long distances. However, this breakthrough hinged not only on the internal combustion engine (ICE) itself but also on the parallel development of a critical support system: gasoline stations. Just as early motorists relied on scattered gas stations to power their journeys, the rise of ICE vehicles necessitated a standardized, accessible network of refueling infrastructure to sustain their adoption. This  relationship between vehicles and their energy infrastructure became a cornerstone of modern transportation, shaping urban planning, economic systems, and global energy policies.


Today, as societies pivot toward sustainability, \acrlongpl{EV} are heralding a similar paradigm shift. Yet their widespread adoption faces a challenge mirroring the early days of automobiles: the need for reliable and efficient charging infrastructure. While \acrlongpl{EV} eliminate tailpipe emissions, their practicality depends on overcoming "range anxiety" and ensuring charging availability aligns with user behavior—issues that gas stations largely resolved for ICE vehicles over a century of iteration. Predicting EV charger usage, therefore, is not merely a technical exercise but a good step in designing infrastructure that mirrors the ubiquity and convenience of gas stations. And helps smoothen transition.
