\pagelayout{wide}
\chapter*{Introduction\;}
\labch{intro}
\todo[inline]{This chapter needs significant development. Remove the "LLM generated slop" note and writing prompt. The introduction should clearly state the research problem, objectives, and significance of the study. It should also provide a brief overview of the methodology and outline the structure of the thesis.}
\textbf{LLM generated slop}
\textit{give historical context on how automobiles came to be and used and how gas stations infrastacturue had to be created. And how EV chargers are very similar to the spread of ICE vehicles.}


The invention of the automobile in the late 19th century revolutionized human mobility, enabling unprecedented freedom to traverse long distances. However, this breakthrough hinged not only on the internal combustion engine (ICE) itself but also on the parallel development of a critical support system: gasoline stations. Just as early motorists relied on scattered fuel depots to power their journeys, the rise of ICE vehicles necessitated a standardized, accessible network of refueling infrastructure to sustain their adoption. This symbiotic relationship between vehicles and their energy infrastructure became a cornerstone of modern transportation, shaping urban planning, economic systems, and global energy policies.

\todo[inline]{This paragraph provides good historical context but needs citations to support the historical claims.}

Today, as societies pivot toward sustainability, \acrfull{EV} are heralding a similar paradigm shift. Yet their widespread adoption faces a challenge mirroring the early days of automobiles: the need for reliable, equitable, and efficient charging infrastructure. While EVs eliminate tailpipe emissions, their practicality depends on overcoming "range anxiety" and ensuring charging availability aligns with user behavior—issues that gas stations largely resolved for ICE vehicles over a century of iteration. Predicting EV charger usage, therefore, is not merely a technical exercise but a good step in designing infrastructure that mirrors the ubiquity and convenience of gas stations. And helps smoothen transition.

\todo[inline]{The introduction needs to be expanded to include: 1) A clear statement of the research problem and questions, 2) The specific objectives of the thesis, 3) The significance and contribution of this research, 4) A brief overview of the methodology, and 5) An outline of the thesis structure. Also, the last sentence "And helps smoothen transition" is grammatically incorrect and should be rewritten.}
