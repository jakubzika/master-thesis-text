\pagelayout{wide}
\setchapterstyle{kao}
% \setchapterpreamble[u]{\margintoc}
\chapter{Conclusion}

In this thesis, we tried to understand the problem of charging demand. For this we have gathered data we tried to utilze existing data to predict the demand. We also tried for the model to have interpretable results.


\section{Practical implications}

Due to the low results the model could not be used in its current state to help planners with the charger placement problem.

We attribute the poor results mainly to the lack of data we provided to the model. This stems from the large variance in data quality and the difficulties we encountered when obtaining the data. And more naive use of spatial processing in regarding of processing the polygons.


\section{Future steps}

\begin{itemize}
    \item focus on studying what factors affect the charging behaviour before building a complex model.
    \item address data bias
    \item incorporate more data (spatial). Like financial. Or identify from what kind of areas people are driving towards the charger. Increase data quantity and quality. This can be used for our model as well as for future models. And potentially more urban prague based models incorporating spatial features.
    \item data quality and centralized storage of the gathered spatial data. Like store in duckdb or postgress with postgis extension. Allowing for easier data manipulations and extraction of features. With stronger aspect on charger data.. So investing time into data pipelines.
    \item consider more simulated approach - large randomness due to stochastic nature of the data, maybe the idea of trying to predict the statistic is not a good one. A simulation framework, with model learning the connection if spatial and temporal features to the prob (That is try to predict some value of a random variable instead of deterministic feature) distribution  of key charging aspects (start time, total lenght of stay, power consumed). With a stronger aspect on trying to identify if the connection even is there. From the learned prob distributions via monte carlo simmulation a charging demand could be obtained. With the addition of uncertainty. With the hope that the model could incorporate uncertainty for areas whose behaviour might simply be too stochastic. The stochastic model could deliver more fine grained data as well as provide predictions for the data our model tried to predict as well. This flexibility would lead to more potential use cases and could capture the inherent stochasticity. Also it is hard to compare the baseline right now due to tho model having to come up with the aggregate statistic for some temporal pattern that is still highly stochastic.
\end{itemize}

% Future research should prioritize understanding the factors driving charging behavior before building complex models. Improving data quality and management through robust pipelines and centralized storage would enable more sophisticated spatial analyses. Given the stochastic nature of charging behavior, a simulation-based approach using probability distributions may prove more effective than deterministic models. Additionally, integrating finer-grained mobility data could reveal more meaningful spatial determinants of charging behavior.
